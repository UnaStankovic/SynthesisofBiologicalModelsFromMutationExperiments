% !TEX encoding = UTF-8 Unicode

\documentclass[a4paper]{article}

\usepackage{color}
\usepackage{url}
\usepackage[T2A]{fontenc} % enable Cyrillic fonts
\usepackage[utf8]{inputenc} % make weird characters work
\usepackage{graphicx}

\usepackage[english,serbian]{babel}
%\usepackage[english,serbianc]{babel} %ukljuciti babel sa ovim opcijama, umesto gornjim, ukoliko se koristi cirilica

\usepackage[unicode]{hyperref}
\hypersetup{colorlinks,citecolor=green,filecolor=green,linkcolor=blue,urlcolor=blue}

%\newtheorem{primer}{Пример}[section] %ćirilični primer
\newtheorem{primer}{Primer}[section]

\begin{document}

\title{Sinteza bioloških modela iz mutacijskih eksperimenata\\ \small{Seminarski rad u okviru kursa\\Verifikacija softvera\\ Matematički fakultet}}

\author{Una Stanković\\una\_stankovic@yahoo.com}
\date{26.~april 2018.}
\maketitle

\abstract{U ovom radu biće izneta osnovna ideja rada pod nazivom "Sinteza bioloških modela iz mutacijskih eksperimenata" (engl. Synthesis of Biological Models from Mutation Experiments) \cite{paper}. }

\newpage

\section{Uvod}
\label{sec:uvod}
Računarsko modelovanje bioloških sistema postaje sve popularnije sa pokušajima naučnika da u potpunosti razumeju kompleksne biološke fenomene.
Razlikuju se dva tipa bioloških problema, u zavisnosti od njihove reprezentacije i potencijala:
\begin{itemize}
\item matematički i
\item računarski
\end{itemize} 
Pristup, u kome od biloških sistema konstruišemo računarske modele, nazivamo "izvršivom biologijom" pošto se ona fokusira na dizajn algoritama, koji imitiraju biološke fenomene, a mogu da se izvršavaju na računaru.\cite{execbio} \\\\
Izvršiva biologija predstavlja novi izazov formalnim metodama.
Naime, javljaju se dva problema sa kojima se susreću biolozi koji se bave ćelijama kada razvijaju modele koji mogu da se analiziraju formalno.\\ Prvi se odnosi na automatsko sintetizovanje konkurentnog $"$in silico$"$ modela (onog koji se izvršava na računaru) za razvoj ćelija sa datim $"$in vivo$"$ eksperimentima (onima koji koriste žive jedinke) u kojima se ispituje kako određene mutacije utiču na ishod eksperimenta. Problem leži u tome što je sinteza, koja se vrši pod mutacijama, jedinstvena, jer mutacije mogu proizvesti nedeterminističke ishode i sintetizovani model mora biti u stanju da ponovi sve te ishode sa ciljem da verno opiše modelovane ćelijske procese. Suprotno tome, običan konkurentni program je korektan ako bira bilo koji ishod koji je predviđen nedeterminističkom specifikacijom. U radu su predstavljeni algoritmi sinteze i sintetizovan je model određivanja sudbine ćelije kod zemljanog crva C. elegans. \\
Drugi se odnosi na problem nedovoljno ograničene specifikacije koji se javlja zbog nepotpunosti setova mutacijskih eksperimenata. Nedovoljno ograničena specifikacija prouzrokuje veći broj različitih modela, koji na različite načine objašnjavaju iste fenomene. Problem ovakve specifikacije odgovara problemu analize prostora zadovoljivih modela. U radu su razvijeni algoritmi koji otkrivaju dvosmislenosti u specifikaciji i uklanjaju redundantnost iz specifikacije, odnosno, kreiraju minimalnu nedvosmislenu specifikaciju.

\subsection{Motivacija}
Bolesti mogu biti uzrokovane izmenama na genu ili mrežama za regulaciju proteina. Uzmimo za primer bolest X koja je usko povezana sa nivoima nekih proteina, na primer, P i R, gde P može negativno regulisati nivo proteina R. Ako bi se desilo da nivo proteina P opadne, došlo bi do povećanja nivoa proteina R što bi mogao biti okidač za bolest X. Da bi se to izbeglo trebalo bi povećati nivo proteina R. Da bi bilo moguće izmeniti mreže regulacije proteina trebalo bi izvesti mutacioni eksperiment, u kome bi ćelije bile genetski modifikovane kako bi se smanjila ili povećala aktivnost određenog proteina. Kao posledica toga ćelija ispoljava abnormalno ponašanje, kao na primer nekontrolisanu deobu. Ako bi, umanjenjem aktivnosti proteina P, rezultujući fenotip ispoljio povećanu aktivnost proteina R, onda bi bilo moguće zaključiti da P negativno reguliše R. Iz ogromnog broja ovakvih eksperimenata biolozi zaključuju regulatorne mreže koje opisuju posledice događaja koji dovode do određenih ćelijskih ponašanja i sudbina.\\\\
Upravo kao posledica svega navedenog, eksperimentalnim biolozima je bitna korektnost modela koje kreiraju kao i detaljno objašenjenje kako je došlo do određenih ishoda. Izvršiva biologija se upravo bavi ovakvim problemima kreiranjem izvršivih modela koji bi bili proverivi nad izvršenim eksperimentima. Tretiranje ćelija kao konkurentnih agenata modeluje činjenicu da ćelije ne evoluiraju istovremeno. Verifikacija obezbeđuje da je konkurentni model korektan za sve varijacije rasta ćelija, ispitivanjem svih mogućih izvršavanja modela.

\begin{primer}
Problem zaustavljanja (eng.~{\em halting problem}) je neodlučiv \cite{haltingproblem}.
\end{primer}

\begin{primer}
Za prevođenje programa napisanih u programskom jeziku C može se koristiti GCC kompajler \cite{gcc}.
\end{primer}

\begin{primer}
 Da bi se ispitivala ispravost softvera, najpre je potrebno precizno definisati njegovo ponašanje \cite{laski2009software}. 
\end{primer}

Reference koje se koriste u ovom tekstu zadate su u datoteci {\em seminarski.bib}. Prevođenje u pdf format u Linux okruženju može se uraditi na sledeći način:
\begin{verbatim}
pdflatex TemaImePrezime.tex 
bibtex TemaImePrezime.aux 
pdflatex TemaImePrezime.tex 
pdflatex TemaImePrezime.tex 
\end{verbatim}
Prvo latexovanje je neophodno da bi se generisao {\em .aux} fajl. {\em bibtex} proizvodi odgovarajući {\em .bbl} fajl koji se koristi za generisanje literature. 
Potrebna su dva prolaza (dva puta pdflatex) da bi se reference ubacile u tekst (tj da ne bi ostali znakovi pitanja umesto referenci). Dodavanjem novih referenci potrebno je ponoviti ceo postupak.  


Broj naslova i podnaslova je proizvoljan. Neophodni su samo Uvod i Zaključak. Na poglavlja unutar teksta referisati se po potrebi. 
\begin{primer}
U odeljku \ref{sec:naslov1} precizirani su osnovni pojmovi, dok su zaključci dati u odeljku \ref{sec:zakljucak}.
\end{primer}

Još jednom da napomenem da nema razloga da pišete:
\begin{verbatim}
\v{s} i \v{c} i \'c ...
\end{verbatim}
Možete koristiti srpska slova
\begin{verbatim}
š i č i ć ... 
\end{verbatim}


Ovde pišem uvodni tekst.
Ovde pišem uvodni tekst. 
Ovde pišem uvodni tekst. 
Ovde pišem uvodni tekst. 


\section{Slike i tabele}
\label{slike_i_tabele}

Slike i tabele treba da budu u svom okruženju, sa odgovarajućim naslovima, obeležene labelom da koje omogućava referenciranje. 

\begin{primer} Ovako se ubacuje slika. Obratiti pažnju da je dodato i 
\begin{verbatim}
\usepackage{graphicx}
\end{verbatim}

\begin{figure}[h!]
\begin{center}
\includegraphics[scale=0.75]{panda.jpg}
\end{center}
\caption{Pande}
\label{fig:pande}
\end{figure}

Na svaku sliku neophodno je referisati se negde u tekstu. Na primer, na slici \ref{fig:pande} prikazane su pande. 
\end{primer}

\begin{primer} I tabele treba da budu u svom okruženju, i na njih je neophodno referisati se u tekstu. Na primer, u tabeli \ref{tab:tabela1} su prikazana različita poravnanja u tabelama.

\begin{table}[h!]
\begin{center}
\caption{Razlčita poravnanja u okviru iste tabele ne treba koristiti jer su nepregledna.}
\begin{tabular}{|c|l|r|} \hline
centralno poravnanje& levo poravnanje& desno poravnanje\\ \hline
a &b&c\\ \hline
d &e&f\\ \hline
\end{tabular}
\label{tab:tabela1}
\end{center}
\end{table}

\end{primer}





\section{Prvi naslov}
\label{sec:naslov1}


Ovde pišem tekst. 
Ovde pišem tekst. 
Ovde pišem tekst. 
Ovde pišem tekst. 
Ovde pišem tekst. 
Ovde pišem tekst. 
Ovde pišem tekst. 
Ovde pišem tekst. 


\subsection{Prvi podnaslov}
\label{subsec:podnaslov1}

Ovde pišem tekst. 
Ovde pišem tekst. 
Ovde pišem tekst. 
Ovde pišem tekst. 
Ovde pišem tekst. 
Ovde pišem tekst. 
Ovde pišem tekst. 

\subsection{Drugi podnaslov}
\label{subsec:podnaslov2}

Ovde pišem tekst. 
Ovde pišem tekst. 
Ovde pišem tekst. 
Ovde pišem tekst. 
Ovde pišem tekst. 
Ovde pišem tekst. 

\section{Drugi naslov}
\label{sec:naslov2}

Ovde pišem tekst. 
Ovde pišem tekst. 
Ovde pišem tekst. 
Ovde pišem tekst. 

\subsection{... podnaslov}
\label{subsec:podnaslovN}

Ovde pišem tekst. 
Ovde pišem tekst. 
Ovde pišem tekst. 
Ovde pišem tekst. 
Ovde pišem tekst. 
Ovde pišem tekst. 

\section{n-ti naslov}
\label{sec:naslovN}

Ovde pišem tekst. 
Ovde pišem tekst. 
Ovde pišem tekst. 
Ovde pišem tekst. 
Ovde pišem tekst. 

\subsection{... podnaslov}
\label{subsec:podnaslovK}

Ovde pišem tekst. 
Ovde pišem tekst. 
Ovde pišem tekst. 
Ovde pišem tekst. 
Ovde pišem tekst. 

\subsection{... podnaslov}
\label{subsec:podnaslovM}

Ovde pišem tekst. 
Ovde pišem tekst. 
Ovde pišem tekst. 
Ovde pišem tekst. 
Ovde pišem tekst. 

\section{Poslednji naslov}
\label{sec:naslovM}

Ovde pišem tekst. 
Ovde pišem tekst. 
Ovde pišem tekst. 
Ovde pišem tekst. 
Ovde pišem tekst. 
Ovde pišem tekst. 
Ovde pišem tekst. 
Ovde pišem tekst. 
Ovde pišem tekst. 

\section{Zaključak}
\label{sec:zakljucak}

Ovde pišem zaključak. 
Ovde pišem zaključak. 
Ovde pišem zaključak. 
Ovde pišem zaključak. 
Ovde pišem zaključak. 
Ovde pišem zaključak. 
Ovde pišem zaključak. 
Ovde pišem zaključak. 
Ovde pišem zaključak. 
Ovde pišem zaključak. 
Ovde pišem zaključak. 
Ovde pišem zaključak. 


\addcontentsline{toc}{section}{Literatura}
\appendix
\bibliography{seminarski} 
\bibliographystyle{plain}

\appendix
\section{Dodatak}
Ovde pišem dodatne stvari, ukoliko za time ima potrebe.
Ovde pišem dodatne stvari, ukoliko za time ima potrebe.
Ovde pišem dodatne stvari, ukoliko za time ima potrebe.
Ovde pišem dodatne stvari, ukoliko za time ima potrebe.
Ovde pišem dodatne stvari, ukoliko za time ima potrebe.


\end{document}
